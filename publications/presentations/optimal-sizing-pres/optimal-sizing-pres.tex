%        File: arfc-beamer.tex
%     Created: Sun May 5 10:00 PM 2013 C
%


%\documentclass[11pt,handout]{beamer}
\documentclass[9pt]{beamer}
\usetheme[white]{Illinois}
%\title[short title]{long title}
\title[Short Title]{Optimal Sizing of a Nuclear Reactor for Embedded Grid Systems}
\subtitle[short subtitle]{ANS National Conference 2020}
% \subtitle[Short SubTitle]{Preliminary Work}
%\author[short name]{long name}
\author[Samuel Dotson]{\textbf{Samuel G. Dotson} and Kathryn D. Huff\\Advanced Reactors and Fuel Cycles Group}
%\date[short date]{long date}
% \date[05.21.2020]{May 21, 2020}
\date{June 10, 2020}
%\institution[short name]{long name}
\institute[UIUC]{University of Illinois at Urbana-Champaign}

%\usepackage{bbding}
\usepackage{amsfonts}
\usepackage{amsmath}
\usepackage{xspace}
\usepackage{graphicx}
\graphicspath{{../../../figures/}}
\usepackage{subfigure}
\usepackage{booktabs} % nice rules for tables
\usepackage{microtype} % if using PDF
\usepackage{bigints}
\usepackage{minted}

\newcommand{\units}[1] {\:\text{#1}}%
\newcommand{\SN}{S$_N$}%{S$_\text{N}$}%{$S_N$}%
\DeclareMathOperator{\erf}{erf}
%I need some complimentary error funcitons...
\DeclareMathOperator{\erfc}{erfc}
%Those icons in the references are terrible looking
\setbeamertemplate{bibliography item}[text]

%%%% Acronym support

\usepackage[acronym,toc]{glossaries}
../acros.tex

\makeglossaries

%try to get rid of header on title page\dots
\makeatletter
    \newenvironment{withoutheadline}{
        \setbeamertemplate{headline}[default]
        \def\beamer@entrycode{\vspace*{-\headheight}}
    }{}
\makeatother

\makeatother
\setbeamertemplate{footline}
{
  \leavevmode%
  \hbox{%
    \rightline{\insertframenumber{} / \inserttotalframenumber\hspace*{1ex}}
  }%
  \vskip0pt%
}
\makeatletter

%%%%%%%%%%%%%%%%%%%%%%%%%%%%%%%%%%%%%%%%%%%%%%%%%%%%%%%%%%%%%
%%%%%%%%%%%%%%%%%%%%%%%%%%%%%%%%%%%%%%%%%%%%%%%%%%%%%%%%%%%%%
% Begin Document
%%%%%%%%%%%%%%%%%%%%%%%%%%%%%%%%%%%%%%%%%%%%%%%%%%%%%%%%%%%%%
%%%%%%%%%%%%%%%%%%%%%%%%%%%%%%%%%%%%%%%%%%%%%%%%%%%%%%%%%%%%%

\begin{document}
%%%%%%%%%%%%%%%%%%%%%%%%%%%%%%%%%%%%%%%%%%%%%%%%%%%%%%%%%%%%%
%% From uw-beamer Here's a handy bit of code to place at
%% the beginning of your presentation (after \begin{document}):
\newcommand*{\alphabet}{ABCDEFGHIJKLMNOPQRSTUVWXYZabcdefghijklmnopqrstuvwxyz}
\newlength{\highlightheight}
\newlength{\highlightdepth}
\newlength{\highlightmargin}
\setlength{\highlightmargin}{2pt}
\settoheight{\highlightheight}{\alphabet}
\settodepth{\highlightdepth}{\alphabet}
\addtolength{\highlightheight}{\highlightmargin}
\addtolength{\highlightdepth}{\highlightmargin}
\addtolength{\highlightheight}{\highlightdepth}
\newcommand*{\Highlight}{\rlap{\textcolor{HighlightBackground}{\rule[-\highlightdepth]{\linewidth}{\highlightheight}}}}
%%%%%%%%%%%%%%%%%%%%%%%%%%%%%%%%%%%%%%%%%%%%%%%%%%%%%%%%%%%%%
%%--------------------------------%%
\begin{withoutheadline}
\frame{
  \titlepage
}
\end{withoutheadline}

%%--------------------------------%%
\AtBeginSection[]{
\begin{frame}
  \frametitle{Outline}
  \tableofcontents[currentsection]
\end{frame}
}

\section{Motivation}
\subsection{Illinois Climate Action Plan (iCAP)}
\input{icap}
% \subsection{Problems}
% \input{cat_math}
\section{Methods}
\section{Methodology}
\gls{temoa} is an open source tool for energy system optimization that
formulates and solves a linear optimization problem
\cite{decarolis_tools_2020}. A linear optimization
problem has two requirements: An objective function and constraints. The
objective function in \gls{temoa} is total system cost over time horizon
of interest and the minimum required constraint is annual demand (and
technology options to meet that demand). Users can optionally add other
constraints to match the real system being modeled. In our case we added
emissions limits based on the carbon goals set by \gls{icap}. At each time step,
\gls{temoa} must be able to meet the various constraints with the existing
capacity, or build be able to build new capacity to do so. If demand and
emissions limits cannot be satisfied, then \gls{temoa} gives ``no solution.''
Mathematically, \gls{temoa} solves the following problem:
\begin{align}
  \intertext{Minimize}
  C_{tot} &= C_{cap} + C_{fix} + C_{var}\\
  \intertext{Subject to:}
  D_i &= \sum_{tech}A_{tech} \mbox{ for } i \in \mbox{ years}
\end{align}

% \subsection{RAVEN}
% \subsection{TEMOA}

% \section{Results}
\section{Grid Characterization: RAVEN}
\input{ravenres}
\section{Optimal Sizing: Temoa}
\input{temoares}
\section{Conclusion}
\section{Conclusion}

In this study we used the \gls{esom} called \gls{temoa} to find the optimal
size of a nuclear reactor for the \gls{uiuc} microgrid. We first showed that
\gls{temoa} gave realistic results that matched predictions from both \gls{icap}
and the \gls{uiuc} Master Plan \cite{isee_illinois_2015, affiliated_engineers_inc_utilities_2015}.
Then we considered three scenarios that introduced nuclear capacity to
\gls{uiuc}. The first two scenarios did not constrain the size of the nuclear
reactor and thus satisfied the carbon constraints and exceeded the steam
and electricity demand requirements by building more nuclear capacity than
required.
The \gls{uiuc} Master Plan found that the goals outlined in \gls{icap} could
not be achieved with \gls{uiuc}'s current energy mix, which we corroborated in
our business-as-usual scenario. We showed in Scenario 3 that the \gls{icap}
goals could be met for the next decade by adding a modest capacity for nuclear
energy production. The assumptions of the model used in this study include
contributions from renewables, but exclude requirements of zero growth,
improvements in building efficiency, and other offsets. This gives \gls{uiuc}
the flexibility to continue growing while reducing carbon emissions in other
areas. The breakdown of carbon offsets shown in Figure \ref{fig:icap_emissions}
is improved by adding nuclear power to the energy mix.
Finally, importing electricity drove the campus carbon emissions in every
scenario we examined. If \gls{uiuc} is serious about decarbonizing by 2050, the
University must stop buying electricity from MISO. Unless, that is, energy
production throughout MISO also becomes carbon free.

Besides producing emissions free electricity and steam, nuclear power can
benefit campuses, like \gls{uiuc}, in many ways. Future work will explore how
nuclear power can help decarbonize campus transportation, the role of energy
storage, and peer further into the future.

\section{Future Work}
\input{futurework}
\section{Acknowledgments}
This work was made possible with the support from the people at \gls{uiuc}
Facilities \& Services. In particular, Morgan White, Mike Marquissee, and Mike
Larson. It was also aided by other members of the \gls{ARFC} group, in
particular, Roberto Fairhurst and David Atwater.
This work is supported by the Nuclear Regulatory Commission Fellowship Program.
Prof. Huff is supported by the Nuclear Regulatory Commission Faculty
Development Program (award NRC-HQ-84-14-G-0054 Program B), the Blue Waters
sustained-petascale computing project supported by the National Science
Foundation (awards OCI-0725070 and ACI-1238993) and the state of Illinois, the
DOE ARPA-E MEITNER Program (award DE-AR0000983), and the DOE H2@Scale Program
(Award Number: DE-EE0008832)

%%--------------------------------%%
%%--------------------------------%%
\begin{frame}[allowframebreaks]
  \frametitle{References}
  \bibliographystyle{plain}
  {\footnotesize \bibliography{bibliography.bib} }

\end{frame}
\begin{frame}
  \begin{center}
    \Huge{\textbf{Questions?}}
  \end{center}
\end{frame}


% Back up slides
\begin{frame}
  \frametitle{Mathematics of \texttt{Temoa}}
  \begin{columns}
    \column[t]{3cm}
    \begin{align}
      \intertext{Minimize}
      C_{total} &= \sum_{t, v}C_{t, v}\\
      \intertext{Subject to}
      D_{s,p} &= \sum_{s,p}G_{s,p}\nonumber\\
      L_{p} &= \sum_{t,p}\hat{R}_{CO_2,\{t,p\}}\nonumber
    \end{align}
    \column[t]{4cm}
    \begin{align}
      \intertext{Where}
      C_{total} &= \text{total cost}\nonumber\\
      D_{s,p} &= \text{energy demand by sector, time period}\nonumber\\
      G_{s,p} &= \text{energy generation by sector, time period}\nonumber\\
      L_{p} &= \text{emission limits by time period}\nonumber\\
      \hat{R}_{CO_2,\{t,p\}} &= \text{emissions by technology, time period}\nonumber
    \end{align}
  \end{columns}
\end{frame}

\begin{frame}
  \frametitle{FAQ}
    \begin{itemize}
      \item \textit{What does "TURBINE" mean?}\\
      The "TURBINE" technology simply converts steam to electricity since Abbott
      power plant is actually a cogeneration plant. We assumed that a nuclear
      reactor that could replace Abbott would also be used for cogeneration.
      \item \textit{Can this analysis be applied to other universities or energy
      systems?}\\
      Yes. While the University of Illinois is unique in its self-reliance, the
      idea that nuclear power fulfills a role in the energy mix that is not easily
      satisfied by renewables is not.
    \end{itemize}
\end{frame}

\begin{frame}
  \frametitle{\#ShutDownSTEM}
  \begin{columns}
    \column[t]{5cm}
    \begin{figure}
      \centering
      \includegraphics[width=\textwidth]{shutdownstem.png}
    \end{figure}
    \column[t]{5cm}
    \vspace{2cm}
    \newline
    % \begin{minipage}[c]{0.6\textheight}
      This lecture has been pre-recorded.\\
      Questions can be directed to:\\
      Samuel G. Dotson\\
      sgd2@illinois.edu\\
      Thank you.
    % \end{minipage}


  \end{columns}
\end{frame}

%%--------------------------------%%


\end{document}
