\section{Model and Data}

The system we modeled in this study is based on the current energy mix of
\gls{uiuc}.

\begin{table*}[ht]
  \centering
  % \caption{A summary of the technologies at \gls{uiuc}}
  % \label{tab:model}
  \begin{tabular}{cc}
    Technology & Name \\
    Natural Gas \& Coal Plant & \texttt{ABBOTT}\\
  \end{tabular}
\end{table*}

\begin{enumerate}
  \item Explain the time horizon
  \begin{itemize}
    \item 2020 is considered a historical year and reflects the current energy
    mix of the university.
    \item The model optimizes years 2021-2030 in single year increments.
    \item In this study, one year is divided 6 time slices. 3 seasons, and 2 times
    of day. Future work will refine this temporal detail.
  \end{itemize}
  \item Typical demand for winter, summer, and the spring/fall ``inter" season
  are determined by averaging historical data from 2015-2018.
  \item The natural gas plant, "ABBOTT" as a cogeneration plant that produces
  all of the steam on campus and much of the electricity. In order to capture
  the cogeneration, we introduced a "TURBINE" technology that produces electricity
  from steam. The proposed nuclear reactor will also produce steam that "TURBINE"
  can use to produce electricity. Thus the model assumes that the nuclear reactor
  will serve as a direct replacement of "ABBOTT" or function alongside "ABBOTT"
  in an identical way.
  \item Natural gas prices.\\
  The price of natural gas is one of main factors driving the choice of energy
  production at UIUC. Since 2014, natural gas prices have somewhat steadily declined.
  \item Carbon Emissions \\
  Carbon emissions in the model are captured by using a carbon emission equivalent
  that matches the strategy adopted by iCAP.
  \item Capacity caps\\
  Solar and wind capacities are both capped by Temoa and reflect the reality of
  the UIUC energy mix.
  \begin{itemize}
    \item The cap on solar energy is due to the maximum capacity of the solar
    farms on campus. Currently, the solar farm is rated to produce 4.68 MWe, but
    will be quadrupled in 2022 when the university finishes the planned Solar Farm 2.0.
    \item The cap on wind energy is due to the 10-year power purchase agreement
    between UIUC and Rail-splitter Wind Farm. This contract ends in 2026, at
    which point the university can elect to purchase more or not.
  \end{itemize}
  \item Offsets, Growth, and Building Standards \\
  This model assumes an energy demand growth of 1\% per year. Thus, offsets like
  shutting down the Blue Waters Supercomputer and improving building standards,
  which serve to reduce demand, are not accounted for and assumes the university
  will carry on with business as usual in every regard except its energy mix.
  \item Scenarios \\
  Describe the modeled scenarios - BAU, 1, 2, 3. \\
  Uncertainty analysis is only performed on scenario 3 because scenarios 1 and 2
  will be pushed along the same technology trajectory because not limiting the
  size of the nuclear reactor means demand and emissions constraints can be
  satisfied arbitrarily. The business as usual scenario is not analyzed for
  uncertainty because it served as a sanity check to verify that Temoa was giving
  appropriate results.
\end{enumerate}
